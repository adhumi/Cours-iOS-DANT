\documentclass[12pt]{article}

\include{preamble}

\newtoggle{spacingmode}
%\toggletrue{spacingmode}  %STUDENTS: DELETE or COMMENT this line

\newtoggle{professormode}
%\toggletrue{professormode} %STUDENTS: DELETE or COMMENT this line

\newcommand{\spc}[1]{\iftoggle{spacingmode}{\\ \vspace{#1cm}}}




\begin{document}
\homework{LI385 - Introduction au d�veloppement iOS}{TP 1 - D�couverte d'Objective C et d'Xcode}{Adrien Humiliere}{11/03/2015}

\noindent

\textit{A la fin de ce TP :
\begin{itemize}
\item Faire une archive contenant les projets Xcode des exercices.
\item Envoyer l'archive � \underline{adrien.humiliere@djit.fr}.
\item Le rendu ne sera pas not� mais fera l'objet d'une correction.
\item Si le TP est fait � plusieurs, pr�ciser les noms et adresses mail de chacun.
\end{itemize}}

\problem{Liste chain�e} L'objectif de l'exercice est d'impl�menter simplement une liste chain�e, uniquement avec les types de base d'Objective-C.

\begin{enumerate}

\subproblem Lancer Xcode et cr�er un nouveau projet de type Command Line Tool (OSX > Application > Command Line Tool).

\subproblem Cr�er une classe \texttt{Node}, avec deux attributs :
\begin{itemize}
\item \texttt{key}, un entier non modifiable
\item \texttt{next}, une r�f�rence vers un objet de type Node
\end{itemize}
Impl�menter � minima la m�thode \texttt{-description} qui, � la fa�on de la m�thode \texttt{toString()} de Java, permet d'afficher l'�tat courant d'un objet dans le terminal.

\subproblem Cr�er une classe List, qui impl�mente une liste simplement chain�e. Cette classe devra impl�menter la m�thode \texttt{-description} et r�pondre aux m�thodes suivantes :

\begin{minted}{objc}
- (BOOL)isEmpty;
- (void)addFirst:(Node *)node; 
- (Node *)removeFirst; 
- (void)addLast:(Node *)node; 
- (Node *)removeLast;
- (Node *)nodeForKey:(int)key;
\end{minted}

\subproblem Tester le programme en utilisant dans la fonction main() chacune des m�thodes impl�ment�es pour la classe List, avec au moins 10 instances de Node dans la liste chain�e.

\end{enumerate}


\problem{Premi�re UI} L'objectif de cet exercice est d'avoir un premier contact avec les classes de Cocoa Touch et avec les outils de cr�ation d'UI fournis par Xcode.

\begin{enumerate}

\subproblem Cr�er un nouveau projet d'application iOS (iOS > Application > Single View Application).

\subproblem Prendre le temps de se familiariser avec les diff�rents fichiers cr��s par Xcode pour un projet de ce type (AppDelegate, ViewController, Storyboard, xib, ...).\\
Utiliser \texttt{main.storyboard} pour ajouter un bouton et deux labels dans l'application.

\subproblem R�cup�rer l'action (\texttt{IBAction}) du bouton pour l'ev�nement \texttt{touchUpInside}, ainsi qu'une r�f�rence (\texttt{IBOutlet}) vers chaque label.\\
Au clic sur le bouton, changer le texte affich� dans les labels.

\subproblem Au clic sur le bouton, afficher la date au moment du clic dans le premier label. Afficher le temps depuis le dernier clic dans le second label.\\
Utiliser la classe \texttt{NSDate}.

\subproblem En utilisant la classe \texttt{NSDateFormatter}, afficher les information de date et heure dans un format facilement lisible par l'utilisateur.

\end{enumerate}

\end{document}