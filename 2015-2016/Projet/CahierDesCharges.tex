\documentclass[12pt]{article}

\include{preamble}

\newtoggle{spacingmode}
%\toggletrue{spacingmode}  %STUDENTS: DELETE or COMMENT this line

\newtoggle{professormode}
%\toggletrue{professormode} %STUDENTS: DELETE or COMMENT this line

\newcommand{\spc}[1]{\iftoggle{spacingmode}{\\ \vspace{#1cm}}}

\begin{document}
\homework{LI385 - Nouvelles Technologies du Web}{Projet - Cahier des charges}{Adrien Humiliere \& Olivier Pitton}{2015/2016}

\noindent

\section*{Application permettant de visualiser ses contacts sur une carte}

Par groupe de 3 ou 4 �tudiants, vous devrez r�aliser une application iOS et le serveur correspondant permettant � un utilisateur de retrouver ses connaissances sur une carte.

La notation portera sur le serveur et le client iOS � 50/50. Les deux parties devront �tres abord�es avec la m�me rigueur. 

\subsection*{Liste des fonctionnalit�s attendues}
\begin{itemize}
\item Afficher dans le client une carte du lieu d�sir�
\item Afficher les emplacements des contacts sur la carte
\item Cr�er des comptes utilisateurs
\item Trouver et ajouter des contacts
\item Activer/d�sactiver le partage de sa position
\item Maintenir la connexion sur le client d?une session � l?autre
\end{itemize}

\subsection*{Consid�rations techniques}
\begin{itemize}
\item Utilisation de Git, selon le workflow �tudi� en cours
\item Client : Swift, iOS 8, iPhone
\item Serveur : Java 7/8 
\item Gestionnaires de d�pendances : Cocoapods (iOS), Maven (Java)
\end{itemize}

\vspace{5mm}

Si, et seulement si, votre avanc�e dans le projet le permet, vous pourrez vous pencher sur des fonctionnalit�s plus pouss�es : connexion via un compte social, localisation en temps r�el, application universelle iPhone/iPad, etc. Si les fonctionnalit�s attendues ne sont pas achev�es, ces plus ne seront pas pris en compte dans la note finale.

\end{document}