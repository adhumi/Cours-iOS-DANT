\documentclass[a4paper,11pt]{scrartcl}
\usepackage[T1]{fontenc}
\usepackage[utf8x]{inputenc}
\usepackage{graphicx}
\usepackage{xcolor}

% \usepackage{tgheros}
% \usepackage[defaultmono]{droidmono}

\usepackage{amsmath,amssymb,amsthm,textcomp}
\usepackage{enumerate}
\usepackage{multicol}
\usepackage{tikz}

\usepackage{menukeys}

\usepackage{geometry}
\geometry{total={210mm,297mm},
left=25mm,right=25mm,%
bindingoffset=0mm, top=20mm,bottom=30mm}

\linespread{1.2}

\newcommand{\linia}{\rule{\linewidth}{0.5pt}}

% my own titles
\makeatletter
\renewcommand{\maketitle}{
\begin{center}
\vspace{2ex}
{\huge \@title}
\vspace{1ex}
\\
\linia\\
\@author \hfill \@date
\vspace{4ex}
\end{center}
}
\makeatother
%%%

% custom footers and headers
\usepackage{fancyhdr}
\pagestyle{fancy}
\lhead{}
\chead{}
\rhead{}
\lfoot{\mytitle}
\cfoot{}
\rfoot{\thepage}
\renewcommand{\headrulewidth}{0pt}
\renewcommand{\footrulewidth}{0.5pt}

%

% code listing settings
\usepackage{listings}
\lstset{
    language=Python,
    basicstyle=\ttfamily\small,
    aboveskip={1.0\baselineskip},
    belowskip={1.0\baselineskip},
    columns=fixed,
    extendedchars=true,
    breaklines=true,
    tabsize=4,
    prebreak=\raisebox{0ex}[0ex][0ex]{\ensuremath{\hookleftarrow}},
    frame=lines,
    showtabs=false,
    showspaces=false,
    showstringspaces=false,
    keywordstyle=\color[rgb]{0.627,0.126,0.941},
    commentstyle=\color[rgb]{0.133,0.545,0.133},
    stringstyle=\color[rgb]{01,0,0},
    numbers=left,
    numberstyle=\small,
    stepnumber=1,
    numbersep=10pt,
    captionpos=t,
    escapeinside={\%*}{*)}
}

% Inline graphics
\usepackage{graphicx,calc}
\newlength\myheight
\newlength\mydepth
\settototalheight\myheight{Xygp}
\settodepth\mydepth{Xygp}
\setlength\fboxsep{0pt}
\newcommand*\inlinegraphics[1]{%
  \settototalheight\myheight{Xygp}%
  \settodepth\mydepth{Xygp}%
  \raisebox{-\mydepth}{\includegraphics[height=\myheight]{#1}}%
}

% Swift
\lstdefinelanguage{swift}
{
  morekeywords={
    func,if,then,else,for,in,while,do,switch,case,default,where,break,continue,fallthrough,return,
    typealias,struct,class,enum,protocol,var,func,let,get,set,willSet,didSet,inout,init,deinit,extension,
    subscript,prefix,operator,infix,postfix,precedence,associativity,left,right,none,convenience,dynamic,
    final,lazy,mutating,nonmutating,optional,override,required,static,unowned,safe,weak,internal,
    private,public,is,as,self,unsafe,dynamicType,true,false,nil,Type,Protocol,
  },
  morecomment=[l]{//}, % l is for line comment
  morecomment=[s]{/*}{*/}, % s is for start and end delimiter
  morestring=[b]" % defines that strings are enclosed in double quotes
}
\definecolor{keyword}{HTML}{BA2CA3}
\definecolor{string}{HTML}{D12F1B}
\definecolor{comment}{HTML}{008400}
\lstset{
  language=swift,
  basicstyle=\ttfamily,
  showstringspaces=false, % lets spaces in strings appear as real spaces
  columns=fixed,
  keepspaces=true,
  keywordstyle=\color{keyword},
  stringstyle=\color{string},
  commentstyle=\color{comment},
}
\lstset{language=Swift}

\begin{document}

\newcommand{\mytitle}{TP 3 - Adaptive User Interfaces}
\title{\mytitle}
\author{Adrien Humilière}
\date{05/04/2019}

\maketitle

\section*{Part 1}

\begin{itemize}
\item Open, build and run (\keys{\cmd+R}) the \textbf{Flashlight} project.
\item Observe the size of the simulator on the screen. Use the menu item \menu[,]{Window,Scale} to adjust the size of the simulator screen       .
\item This is a really basic application template. Use it to discover Xcode interface anatomy.
\item Use \keys{\cmd+\tab} and \keys{\cmd+.} to switch to the simulator and back; and to stop the app from Xcode.
\end{itemize}

\section*{Part 2}

\begin{itemize}
\item Open and run (\keys{\cmd+R}) the \textbf{WordCollage} project.
\item Using the Project Navigator (\keys{\cmd+1}), explore \textbf{Main.storyboard}.
\item Using the Show Document Outline control (\inlinegraphics{../assets/show-document-outline.png}) in the lower left corner of the canvas, ensure that the document outline is visible.
\item Double-click a Label in the collage to change its contents.
\item Run the app (\keys{\cmd+R}), and witness the change in the iOS Simulator.
\item Experiment with changing the content of the remaining labels to topics you care about.
\item Run the app (\keys{\cmd+R}), and witness the changes in the Simulator.
\end{itemize}

\section*{Part 3}

\begin{itemize}
\item Use the Project Navigator (\keys{\cmd+1}) to select Main.storyboard.
\item Run the app (\keys{\cmd+R}), and observe how the visual layout of the collage appears different in the iOS Simulator.
\item Using the Object Library (\keys{\Alt+\cmd+L}), place a new Label on the interface. Change the Label contents (e.g. "Learn to Code") and use the Attributes Inspector (\keys{\Alt+\cmd+4}) to change the font family, size and color (e.g. 51pt Avenir Next Ultra Light).
\item Use the Label handles to expand its size, and adjust the Label position.
\item Run the app (\keys{\cmd+R}), and observe how the Label position appears differently in the iOS Simulator.
\item Position constraints must be added to the Label to influence its position.
\item With the Label selected, use the Pin control to select a Vertical Space constraint relative to the View.
\begin{figure}[h]
	\begin{center}
   \includegraphics[width=200pt]{pin_constraint.png}
	\end{center}
\end{figure}
\item Interface Builder displays a vertical blue bar representing the Vertical Space constraint. Missing constraints result in Interface Builder displaying Auto Layout issues in orange.
\item With the Label selected, use the Align control to select a Center X Alignment constraint based on the current position of the Label.
\begin{figure}[h]
	\begin{center}
   \includegraphics[width=200pt]{align_constraint.png}
	\end{center}
\end{figure}
\item Interface Builder displays another vertical blue bar representing the Center X Alignment constraint.
\item Using the Show Document Outline control (\inlinegraphics{../assets/show-document-outline.png}) in the lower left corner of the canvas, ensure that the document outline is visible.
\item Interface Builder displays one remaining Auto Layout issue in orange. Use the Issue Navigator (\keys{\cmd+4}) or the Document Outline disclosure arrow (\inlinegraphics{../assets/disclosure_arrow.png}) to observe the details of the remaining Auto Layout issue.
\item With the Label selected, use the menu item \menu[ > ]{Editor > Resolve Auto Layout Issues > Update Frames} so the frame matches the constraint. Alternatively, use the menu item \menu[ > ]{Editor > Resolve Auto Layout Issues > Update Constraints} so the constraints match the frame.
\item Run the app (\keys{\cmd+R}) and observe how the Label appears in a better position, but still appears somewhat different.
\item Within the Interface Builder canvas, select the recently added Label, adjust its position, update the constraints (\keys{\Alt+\cmd+=}), and observe how the preview automatically reflects the change.
\item Run the app (\keys{\cmd+R}) and observe how the Label appears as expected within the iOS Simulator.
\item Rotate the app (\keys{\cmd+\arrowkeyright}) within the iOS Simulator, and observe how the label appears in a different position when in a landscape orientation.
\item Select the recently added Label, adjust its position, update the constraints (\keys{\shift+\cmd+=}).
\item Run the app (\keys{\cmd+R}), rotate the app (\keys{\cmd+\arrowkeyright}) in the Simulator, and observe the Label appearing in the expected position.
\end{itemize}

\section*{Part 4}

\begin{itemize}
\item Using Interface Builder and the Object Library (\keys{\Alt+\cmd+L}) to place a Button on the interface.
\item With the button selected, discover the Identity (\keys{\Alt+\cmd+3}), Attributes (\keys{\Alt+\cmd+4}) and Size (\keys{\Alt+\cmd+5}) Inspectors.
\item Using Interface Builder, change the text of the button to "Change Background."
\item Run the app (\keys{\cmd+R}) and observe how the button appears in a different location within the iOS Simulator.
\item Using Interface Builder, Control-drag from the Button downward to the View, and select Bottom Space to Bottom Layout Guide to create a Vertical Space constraint.
\item With the Button still selected, use the Align control and select Horizontal Center in Container to create a Center X Alignment constraint.
\item Run the app (\keys{\cmd+R}), tap the button, and observe that nothing happens.
\item While viewing the storyboard in Interface Builder, open the Assistant Editor (\keys{\Alt+\cmd+\return}).
\item Using the Show Document Outline control (\inlinegraphics{../assets/show-document-outline.png}) in the lower left corner of the canvas, ensure that the document outline is visible.
\item Using the Document Outline, Control-click the button and drag a connection from the Touch Up Inside connection well to the controller, to create an Action connection. Use the name \texttt{changeBackgroundColor} and the Type \texttt{UIButton}.
\begin{lstlisting}
@IBAction func changeBackgroundColor(sender: UIButton) {

}
\end{lstlisting}
\item Drawing attention to the connection well next to the method, explain the how Interface Builder relies on the \texttt{@IBAction} attribute to establish connections between interface components and controller code.
\item Experiment with removing the \texttt{@IBAction} attribute, and witness the connection well disappear. Undo the change, and witness the connection well reappear 
\item Implement the \texttt{changeBackgroundColor:} method.
\begin{lstlisting}
@IBAction func changeBackgroundColor(sender: UIButton) {
	view.backgroundColor = UIColor.blackColor()
}
\end{lstlisting}
\item Using the Xcode Documentation and API Reference (\keys{\shift+\cmd+0}), discover the documentation for \texttt{UIColor} to discover other "easy" colors.
\item Run the app (\keys{\cmd+R}), tap the button, and witness the background color change.
\end{itemize}

\section*{Part 5}

\begin{itemize}
\item Change the label of the existing Button contents to "Black."
\item Using Interface Builder and the Object Library (\keys{\Alt+\cmd+L}), add a Button to the bottom left of the interface, labeled "White."
\item Using Interface Builder, Control-drag from the Button downward to the View, and select Bottom Space to Bottom Layout Guide to create a Vertical Space constraint.
\item With the Button still selected, use the Align control and select Horizontal Center in Container using the Current Canvas Value to create a Center X Alignment constraint.
\begin{figure}[h]
	\begin{center}
   \includegraphics[width=200pt]{align_constraint_lesson_4.png}
	\end{center}
\end{figure}
\item Add another button, labeled "Magenta," to the bottom right of the interface, and add constraints similar to the previous Button.
\item Using Interface Builder and the Assistant Editor (\keys{\Alt+\cmd+\return}), establish connections between each button and two new controller methods, \texttt{changeBackgroundColorToWhite:} and \texttt{changeBackgroundColorToMagenta:}.
\begin{lstlisting}
@IBAction func changeBackgroundColorToWhite(sender: UIButton) {
}

@IBAction func changeBackgroundColorToMagenta(sender: UIButton) {
}
\end{lstlisting}
\item Implement the two methods.
\begin{lstlisting}
@IBAction func changeBackgroundColorToWhite(sender: UIButton) {
	view.backgroundColor = UIColor.whiteColor()
}

@IBAction func changeBackgroundColorToMagenta(sender: UIButton) {
	view.backgroundColor = UIColor.magentaColor()
}
\end{lstlisting}
\item Rename \texttt{changeBackgroundColor:} to \texttt{changeBackgroundColorToBlack:}, and observe that the adjacent connection well appears hollow.
\item Run the app (\keys{\cmd+R}), tap the Black button, and witness the app crash. Stop the app (\keys{\cmd+.}).
\item The app crashed because Interface Builder still tries to connect the button to the \texttt{changeBackgroundColor:} method, which no longer exists.
\item Using Interface Builder and the connection overlay, delete the old connection, establish a new connection to \texttt{changeBackgroundColorToBlack:}, and observe the connection well reappear.
\item Run the app (\keys{\cmd+R}), tap the buttons and witness the background color changing.
\end{itemize}


\section*{Part 6}

\begin{itemize}
\item Create a new project, of the kind \textbf{Single View Application}.
\item Using Interface Builder and the Object Library (\keys{\Alt+\cmd+L}), add a text label for the converted temperature.
\item Adjust the auto layout constraints of the label. It should be at a fixed distance of the top, and horizontally centered.
\item Using Interface Builder and the Object Library, add a Picker View to the bottom of the interface and adjust its constraints.
\item Run the app (\keys{\cmd+R}) and attempt to use the picker.
\end{itemize}

\section*{Part 7}

\begin{itemize}
\item The picker view allow the user to pick a text or a date within defined choices. It uses the data source and delegate patterns.
\item Using Interface Builder, set the main View Controller as the picker view datasource by Control-clicking on the picker view, and dragging a connection from the dataSource connection well to the View Controller in the Document Outline (\inlinegraphics{../assets/show-document-outline.png}).
\item Run the app, observe the crash, and inspect the console output.
\item The picker view's data source is the view controller, but the ViewController class does not yet implement the methods that conform to the \texttt{UIPickerViewDataSource} protocol.
\item Using the Xcode Documentation and API Reference (\keys{\shift+\cmd+0}), explore the \\\texttt{UIPickerViewDataSource} Protocol Reference and the methods \texttt{numberOfComponentsInPickerView:} and \texttt{pickerView:numberOfRowsInComponent:}.
\item Add the \texttt{UIPickerViewDataSource} protocol declaration to the controller class.
\begin{lstlisting}
class ViewController: UIViewController, UIPickerViewDataSource {
\end{lstlisting}
\item Open the Issue Navigator (\keys{\cmd+4}), and notice the warnings indicating the methods necessary for conforming to the \texttt{UIPickerViewDataSource} protocol.
\item Implement \texttt{numberOfComponentsInPickerView:} and \texttt{pickerView:numberOfRowsInComponent:}.
\begin{lstlisting}
func numberOfComponentsInPickerView(pickerView: UIPickerView) -> Int {
	return 1
}

func pickerView(pickerView: UIPickerView, numberOfRowsInComponent
	component: Int) -> Int {
	return 10
}
\end{lstlisting}
\item Run the app, and observe that the picker has one scrollable element that contains ten rows.
\end{itemize}

\section*{Part 8}

\begin{itemize}
\item Observe how the picker view displays the \texttt{?} character. Without a delegate to determine what to display, the picker view renders a \texttt{?} by default.
\item Using Interface Builder, set the main View Controller as the picker view delegate by Control-clicking the picker view, and dragging a connection from the delegate connection well to the View Controller in the Document Outline.
\item Add the \texttt{UIPickerViewDelegate} protocol declaration to the controller class.
\begin{lstlisting}
class ViewController: UIViewController, UIPickerViewDataSource, UIPickerViewDelegate {
\end{lstlisting}
\item Using the Xcode Documentation and API Reference, explore the \texttt{UIPickerViewDelegate} Protocol Reference and the methods \texttt{pickerView:titleForRow:forComponent:} and \\\texttt{pickerView:didSelectRow:inComponent:}.
\item In the \texttt{ViewController} class, implement \texttt{pickerView:titleForRow:forComponent:}.
\begin{lstlisting}
func pickerView(pickerView: UIPickerView, titleForRow row: Int, forComponent component: Int) -> String? {
	return "N Celsius degrees"
	}
\end{lstlisting}
\item In the \texttt{ViewController} class, implement \texttt{pickerView:didSelectRow:inComponent:}.
\begin{lstlisting}
func pickerView(pickerView: UIPickerView, didSelectRow row: Int, inComponent component: Int) {
	// convert and display temperature
	}
\end{lstlisting}
\item Add a custom breakpoint to \texttt{pickerView:didSelectRow:inComponent:} that generates a Log message containing \texttt{selected: @row@}.
\begin{figure}[h]
	\begin{center}
   \includegraphics[width=200pt]{breakpoint.png}
	\end{center}
\end{figure}
\item Run the app, observe the values displayed in the picker view, flick the picker to select a row, and observe the console message when the row is selected.
\end{itemize}

\section*{Part 9}

\begin{itemize}
\item We want a "list" of negative and positive Celsius temperatures for the picker view to display, considering the total number of values (how many possible temperatures?), and the range (what minimum and maximum temperatures?).
\item In the controller, add a private property for an \texttt{Array} of temperature values that the controller will provide to the picker view for display.
\begin{lstlisting}
private var temperatureValues = [Int]()
\end{lstlisting}
\item Implement a naive, temporary assignment of the \texttt{temperatureValues} property during \texttt{viewDidLoad}.
\begin{lstlisting}
override func viewDidLoad() {
	super.viewDidLoad()
	temperatureValues = [1, 2, 3, 4, 5]
}
\end{lstlisting}
\item Update the implementation of \texttt{pickerView:titleForRow:forComponent:}.
\begin{lstlisting}
func pickerView(pickerView: UIPickerView, titleForRow row: Int, forComponent component: Int) -> String? {
	let celsiusValue = temperatureValues[row]
	return "\(celsiusValue) celsius degrees"
}
\end{lstlisting}
\item Run the app, observe the values displayed in the picker, and flick the picker one row at a time until the app crashes. Observe the console error.
\item Update \texttt{pickerView:titleForRow:forComponent:} to use the size of the \texttt{temperatureValues} array to inform the picker of how many rows to display.
\item Run the app, observe the temperature values, and interact with the picker.
\item Instead of an explicit array initialization([-100, -99, …, 99, 100]), a programmatic initialization using a loop will be used. Modify \texttt{viewDidLoad} to naively populate the temperatureValues array with a loop.
\begin{lstlisting}
override func viewDidLoad() {
	super.viewDidLoad()
	let lowerBound = -100
	let upperBound = 100
	for index in lowerBound...upperBound {
		temperatureValues.append(index)
	}
}
\end{lstlisting}
\item \texttt{map} might be used to transform a range into an array of \texttt{Int} values.
\item Using the documentation of \texttt{map}, update the \texttt{temperatureValues} property declaration and remove the procedural temperature value generation from \texttt{viewDidLoad}.
\item Run the app, and observe the temperature values in the picker.
\end{itemize}

\section*{Part 10}

\begin{itemize}
\item We now want to convert and display the temperature when selected in the picker view.
\item Using Interface Builder and the Assistant Editor (\keys{\Alt+\cmd+\return}), create an outlet connection for the label as a controller property.
\begin{lstlisting}
@IBOutlet weak var temperatureLabel: UILabel!
\end{lstlisting}
\item Update the ViewController \texttt{pickerView:didSelectRow:inComponent:} method. It will perform the conversion $farenheit=1.8\times celsius+32$ and update \texttt{temperatureLabel} with the value.
\item Run the app, select a temperature with the picker, and observe the displayed temperature.
\item We need to convert the converted temperature to an \texttt{Int} before updating the \texttt{UILabel} text property.
\end{itemize}

\section*{Part 11}

\begin{itemize}
\item The design pattern MVC (Model – View – Controller) is widely used in iOS development. It helps separating code logic from display. In this project, we will try to create a model taking care of the temperature conversion, to extract this logic from the view controller.
\item Add a new Swift class to the project for a UnitConverter model.
\begin{lstlisting}
import Foundation

class UnitConverter {

}
\end{lstlisting}
\item The temperature conversion code in the controller \texttt{pickerView:didSelectRow:inComponent:} method belongs in the model. Add a \texttt{degreesFahrenheit:} method to the \texttt{UnitConverter} class. It should take Celsius degrees in arguments (integers) and return Fahrenheit degrees (integers).
\item In the \texttt{ViewController} class, declare a new private property for a \texttt{UnitConverter} object and affect \texttt{UnitConverter()}.
\item Update the \texttt{pickerView:didSelectRow:inComponent:} method to use the \texttt{UnitConverter degreesFahrenheit:} method.
\end{itemize}

\section*{Part 12}

\begin{itemize}
\item The range of temperature values has nothing to do with unit conversion, and therefore the UnitConverter model should not be responsible of generating a range of temperatures. We will create a "view model": a model object whose sole purpose is to serve the view.
\item Add a new Swift class to the project for a \texttt{TemperatureRange} model.
\item We will now establish a \texttt{TemperatureRange} object as the picker view's \texttt{dataSource}. \texttt{TemperatureRange} will have to adopt the \texttt{UIPickerViewDataSource} protocol, and the picker view's \texttt{dataSource} connection will have to change to a new TemperatureRange object.
\item Using Interface Builder and the Object Library, drag an Object to the View Controller Scene in the Document Outline (\inlinegraphics{../assets/show-document-outline.png}). Rename the Object to \texttt{TemperatureRange}.
\begin{figure}[h]
	\begin{center}
   \includegraphics[width=100pt]{temperaturerange.png}
	\end{center}
\end{figure}
\item With the \texttt{TemperatureRange} object selected, use the Identity Inspector (\keys{\Alt+\cmd+3}) to set the Class to \texttt{TemperatureRange}.
\begin{figure}[h]
	\begin{center}
   \includegraphics[width=100pt]{identityinspector.png}
	\end{center}
\end{figure}
\item Using Interface Builder, select the picker view and use the Connections Inspector (\keys{\Alt+\cmd+6}) to delete the \texttt{dataSource} connection between the picker view and the controller. Drag a new connection from the picker view's dataSource to the TemperatureRange object in the Document Outline.
\item Run the app, observe the crash, and inspect the error displayed in the console. \texttt{TemperatureRange} should implement \texttt{UIPickerViewDataSource}.
\end{itemize}

\section*{Part 13}

\begin{itemize}
\item Change the \texttt{TemperatureRange} class import statement to provide access to the \\\texttt{UIPickerViewDataSource} type.
\begin{lstlisting}
import UIKit
\end{lstlisting}
\item Update the \texttt{TemperatureRange} class to inherit from \texttt{NSObject} and to adopt the \\\texttt{UIPickerViewDataSource} protocol. Remove the \texttt{UIPickerViewDataSource} protocol adoption from the \texttt{ViewController} class definition.
\begin{lstlisting}
class TemperatureRange: NSObject, UIPickerViewDataSource
class ViewController: UIViewController, UIPickerViewDelegate
\end{lstlisting}
\item Move the \texttt{temperatureValues} property out of the controller and into the \texttt{TemperatureRange} class. Remove the private access control modifier, and shorten its name to \texttt{values}.
\begin{lstlisting}
let values = (-100...100).map { $0 }
\end{lstlisting}
\item Move the controller methods \texttt{numberOfComponentsInPickerView:} and \\\texttt{pickerView:numberOfRowsInComponent:} into the \texttt{TemperatureRange} class, and replace the reference to \texttt{temperatureValues} with \texttt{values}.
\begin{lstlisting}
func numberOfComponentsInPickerView(pickerView: UIPickerView) -> Int {
	return 1
}

func pickerView(pickerView: UIPickerView,
	numberOfRowsInComponent component: Int) -> Int {
	return values.count
}
\end{lstlisting}
\item Open the \texttt{ViewController} class, and observe the red error indicators.
\item Using Interface Builder and the Assistant Editor, Control-drag an outlet connection from the \texttt{TemperatureRange} object to the controller class, to create a new property.
\begin{lstlisting}
@IBOutlet var temperatureRange: TemperatureRange!
\end{lstlisting}
\item Update the controller methods \texttt{pickerView:titleForRow:forComponent:} and \\\texttt{pickerView:didSelectRow:inComponent:} to use the new \texttt{temperatureRange} property, replacing references to \texttt{temperatureValues} with \texttt{temperatureRange.values}.
\item Run the app, select a temperature, and observe the converted value.
\item The remaining controller code only manages communication between the view and the models, and updates the view.
\end{itemize}

\section*{Part 14}

\begin{itemize}
\item We want to improve the user experience when first starting the app. Notice the default starting temperature in the picker view, and consider how it affects the user experience. We might implement the behavior of specifying a default starting temperature.
\item Using Interface Builder and the Assistant Editor, add the picker view as an outlet property within the \texttt{ViewController} class.
\begin{lstlisting}
@IBOutlet weak var celsiusPicker: UIPickerView!
\end{lstlisting}
\item Set the default selected temperature in \texttt{viewDidLoad} with the method \\\texttt{selectRow:inComponent:animated:}.
\item Run the app and notice that, while the selected Celsius temperature has changed, the converted temperature label has not updated. Make sure the label is updated in \texttt{viewDidLoad}.
\item Run the app, observe the default selected temperature in the picker, and observe the converted temperature label.
\end{itemize}

\section*{Part 15}

\begin{itemize}
\item Run the app, select a temperature, background the app, then foreground the app. Notice how the last selected temperature is still displayed.
\item Using the multitasking bar (\keys{\shift+\cmd+H} twice quickly), quit the app, then start the app. The app "forgets" the last selected temperature, and displays the default temperature in the picker view. We want the app to "remember" the last selected temperature, and to use that temperature when it starts, if a last-selected temperature is known.
\item \texttt{NSUserDefaults} allow developers to save data in the filesystem as property list (.plist) files.
\item Enhance \texttt{pickerView:didSelectRow:inComponent:} to save the picker view's selected row index.
\begin{lstlisting}
let defaults = NSUserDefaults.standardUserDefaults()
defaults.setInteger(row, forKey: "defaultCelsiusPickerRow")
defaults.synchronize()
\end{lstlisting}
\item The controller method \texttt{pickerView:didSelectRow:inComponent:} now has two responsibilities: updating the temperature label and saving the last-selected row. Extract the code for each respective task into two separate, well-named controller methods.
\begin{lstlisting}
func displayConvertedTemperatureForRow(row: Int)
func saveSelectedRow(row: Int)
\end{lstlisting}
\item Update \texttt{pickerView:didSelectRow:inComponent:} to call the two new methods.
\item Extract the buried string into a constant placed near the top of the \texttt{ViewController} class. Use this constant in \texttt{saveSelectedRow:}.
\begin{lstlisting}
let userDefaultsLastRowKey = "defaultCelsiusPickerRow"
\end{lstlisting}
\item Run the app, select a temperature. Using the multitasking bar, quit the app, start the app again, and observe that, despite saving the last selected picker row, the default row is selected.
\end{itemize}

\section*{Part 16}

\begin{itemize}
\item Refactor \texttt{viewDidLoad} to use an \texttt{initialPickerRow} method, instead of a local variable, to determine the initial selected row index of the picker view.
\begin{lstlisting}
override func viewDidLoad() {
	super.viewDidLoad()
	let row = initialPickerRow()
	celsiusPicker.selectRow(row, inComponent: 0, animated: false)
	pickerView(celsiusPicker, didSelectRow: row, inComponent: 0)
}

func initialPickerRow() -> Int {
	// load from user defaults
	// if we obtained a last-known row index, return it
	// otherwise, return the default.
	return celsiusPicker.numberOfRowsInComponent(0) / 2
}
\end{lstlisting}
\item Using the Xcode documentation and API Reference, explore the \texttt{NSUserDefaults integerForKey:} method, and observe how it returns \texttt{0} when a value for the provided key is not found.
\item Implement a functional version of the \texttt{initialPickerRow} method, that fetch the corresponding data from \texttt{NSUserDefaults} and fallback to the default value if needed.
\item Run the app, select a temperature, force quit the app via the multitasking bar, restart the app, and witness the last selected temperature is correctly displayed.
\end{itemize}

\end{document}
